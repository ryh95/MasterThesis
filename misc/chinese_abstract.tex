	
\begin{chineseabstract}
作为一种重要的数据挖掘方式,聚类在过去的几十年间一直被广泛研究,其无监督的特性在大数据时代备受青睐。在所有聚类问题中,$k$-means问题和谱聚类问题可能是最为知名的两个问题。然而,这两个问题都是NP难的,因此,在大数据下我们只能使用近似算法。
如何改进这两个问题上的解的质量以及更快获得近似解则是两个关键问题,本篇文章将对$k$-means和谱聚类分不同章节按照以下方式对这两个问题进行研究。首先,在对应的章节,文章将分别定义这两个聚类问题并
介绍相关背景;然后,从改进聚类质量和加速两个方面分别介绍国内外先进研究成果;接着,阐述我们的贡献和创新;随后,用实验验证相关算法和理论;最后, 总结现有的成果并对未来的方向予以展望和预测。
% 这个问题易于理解,相关算法较易实现,是数据分析方面的经典。另一方面,由于$k$-means只能将数据线性分割,谱聚类问题因此被提了出来,借由谱分解的手段和更加灵活的相似度定义,原始空间中的数据在特征空间中变得更容易分割,因此它也是当前大数据分析的研究热点之一。


本人在研究$k$-means和谱聚类的算法和理论的基础上,对相关算法和理论做了进一步改进,主要工作如下:

1. 对基于均匀采样加速$k$-means的算法给出了更紧的理论界,证明了在温和的假设下该算法的高效性,其时间复杂度在多项式对数级别。

2. 将经典的$k$-means++扩展到了带权重的$k$-means问题上并给出了证明。

3. 对基于带权kernel $k$-means的谱聚类算法给出了更紧的理论界。

4. 在$k$-means上对均匀采样、K-M$\text{C}^2$、Double-K-M$\text{C}^2$和它们的kernel版给出了MATLAB实现,实验验证了这些算法的效率和有效性。

5. 用MATLAB实现了基于带权kernel $k$-means的谱聚类算法和它对应的均匀采样版,实验说明了均匀采样版本的高效和合理的聚类质量。

我们的工作不仅在理论上证明了基于均匀采样算法的解的质量,而且在实验上验证了这一算法聚类质量和效率的良好平衡性。和seeding算法相比,均匀采样带来了显著的聚类效果的提升,时间花费又不会太多,和不采样相比,聚类质量虽有少量下降,但是速度却快了几十倍。在未来,这一算法可以广泛应用在大数据分析上,比如大型数据库去重、大型社区检测、海量文本挖掘等。

\chinesekeyword{$k$-means,采样,理论保证,谱聚类,Nyström方法}
\end{chineseabstract}

