	
\begin{chineseabstract}
随着信息技术的发展,移动互联网和物联网等领域正经历着迅猛的发展,海量的数据伴随产生,在海量数据
下挖掘数据的模式正变得日益迫切与重要。作为一种重要的数据挖掘方式,聚类在过去的几十年间一直被广泛研究,其无监督的特性在大数据时代备受青睐。在所有聚类问题中,$k$-means问题可能是最为知名的问题,这个问题易于理解,相关算法较易实现,是大数据分析方面的经典。另一方面,由于求解$k$-means只能将数据线性分割,谱聚类问题因此被提了出来,借由更加自由的相似度定义和谱分解的手段,原始空间中的数据在特征空间中变得更容易分割。得益于其优秀的聚类效果,谱聚类也是当前大数据分析的研究热点之一。

遗憾的是,这两个问题都是NP难的,所以,目前在大数据下,我们只能使用近似算法。
在海量数据下,如何改进传统算法在这两个问题上的解以及更快获得近似解则是两个关键问题,本篇文章将对$k$-means和谱聚类分不同章节按照以下方式对这两个问题进行研究。首先,在对应的章节,文章将分别定义这两个聚类问题并
介绍相关背景;然后,从改进聚类质量和加速两个方面分别介绍国内外先进研究成果;接着,阐述我们的贡献和创新;随后,用实验验证相关算法和理论;最后, 总结现有的成果并对未来的方向予以展望和预测。

本人在研究$k$-means和谱聚类的算法和理论的基础上,对相关算法和理论做了进一步改进,主要工作如下:

1. 对基于均匀采样加速$k$-means的算法给出了更紧的理论界,证明了在温和的假设下,该算法的时间复杂度在多项式对数级别,用实验验证了理论。

2. 将经典的$k$-means++扩展到了带权重的$k$-means问题上并给出了证明。

3. 对基于带权kernel $k$-means的谱聚类算法给出了更紧的理论界,并给出了带权kernel $k$-means的MATLAB实现。

我们的工作不仅在理论上证明了基于均匀采样算法的解的质量,而且在实验上验证了这一算法聚类质量和效率的良好平衡性。和seeding算法相比,均匀采样带来了显著的聚类效果的提升,时间花费又不会太多,和不采样相比,聚类质量虽有少量下降,但是速度却快了几十倍。在未来,这一算法可以广泛应用在大数据分析上,比如大型数据库去重、大型社区检测、海量文本挖掘等。

\chinesekeyword{$k$-means,采样,理论保证,谱聚类,Nyström方法}
\end{chineseabstract}

