	
\begin{chineseabstract}
随着信息技术的发展,移动互联网和物联网等领域正经历着迅猛的发展,海量的数据伴随产生,在海量数据
下挖掘数据的模式正变得日益迫切与重要。聚类作为一种重要的数据挖掘方式,其在过去的几十年间一直被广泛研究。在所有聚类问题中,$k$-means问题和谱聚类问题无疑是两个知名的问题。
如何改进传统算法在这两个问题上的解以及在海量数据下更快获得一个近似解则是两个关键问题,本篇文章将对$k$-means问题和谱聚类问题分不同章节按照以下方式对这两个问题进行研究。首先,在对应的章节,文章将分别定义这两个聚类问题并
介绍相关背景;然后,从改进聚类质量和加速两个方面分别介绍国内外先进研究成果;接着,阐述我们的贡献和创新;随后,用实验验证相关算法和理论;最后, 总结现有的成果并对未来的方向予以展望和预测。

\chinesekeyword{聚类,$k$-means,采样,次线性时间算法,理论保证,谱聚类,Nyström方法,landmark方法,谱旋转,带权kernel $k$-means}
\end{chineseabstract}

