
\begin{englishabstract}
	With the rapid development of the information technology, areas like Mobile Internet and Internet of Things are
growing fast. Driven by these areas, more and more data has been generated and large amounts of data need to be processed.
Hence, data mining is becoming more and more important under this big data scenario. As a crucial data mining technique,
clustering has been studied over the past few decades. Among all the clustering problems, the $k$-means and the spectral clustering are two well-known problems undoubtedly. How to improve the clustering quality and speedup the clustering procedure on the big data are two key problems. This thesis will investigate these two problems on the $k$-means and the spectral clustering in different chapters in the following ways. First, the paper will define the $k$-means or spectral clustering in the corresponding chapters and introduce relevant backgrounds. Second, techniques for improving the clustering quality and speeding up will be introduced separately and described in details. Third, our contribution and innovation will be emphasized. After that, experiments will be caried out to verify our algorithms and theoretical conclusions. Finally, the main results will be summarized and the future directions will be forecasted.

	\englishkeyword{clustering, $k$-means, sampling, sub-linear time algorithms, theoretical guarantee, spectral clustering, Nyström method, landmark-based method, spectral rotation, weighted kernel $k$-means}
\end{englishabstract}


