\section{本章小结}
至此,本章梳理了在$k$-means问题上的加速算法,有理论保证的算法和两者都有的算法,这些算法的相关结论总结在表\ref{tab: kmeans_algorithms}中,其中$O(\mathcal{A}_c)$是$\alpha$近似算法的时间复杂度。根据表\ref{tab: kmeans_algorithms},我们对这些算法分别从速度、理论保证和实用性这三个维度进行评价。这些算法按照时间复杂度从小到大排序如下:K-M$\text{C}^2$、均匀不放回采样(2)、AFK-M$\text{C}^2$、Lightweight Coreset、$k$-means++、算法\ref{alg: reduce_k-means2}(1)、$\epsilon$-Coreset、$k$-means++\&swap、$k$-means\(\vert \vert\) seeding、算法\ref{alg: reduce_k-means2}(2)和p-swap。在排序中,1、2名是次线性时间(需要假设),3、4名在$O(nd)$量级,5、6、7名在$O(nkd)$量级,最后4名从$O(ndk^2)$逐渐增加到$O(n^3)$。这些算法中后6名有常数近似,$k$-means++是$O(\log k)$,而前4名不是要假设就是要加方差项。最后一个,算法实用与否也非常重要,从实用角度讲,这些算法可以分为3类,第一类是AFK-M$\text{C}^2$、Lightweight Coreset、$\epsilon$-Coreset、$k$-means\(\vert \vert\) seeding,这些算法并行性较好,也无需假设,非常实用;第二类是K-M$\text{C}^2$、均匀不放回采样(2)、$k$-means++、算法\ref{alg: reduce_k-means2}(1)、算法\ref{alg: reduce_k-means2}(2)、$k$-means++\&swap,这些算法不是要假设就是有强序列属性不利于并行,实用性一般;最后剩下的就是p-swap,时间复杂度高,不实用。总结来说,如果追求速度推荐使用K-M$\text{C}^2$、均匀不放回采样(2)、AFK-M$\text{C}^2$和Lightweight Coreset,如果追求聚类质量推荐使用$\epsilon$-Coreset和$k$-means\(\vert \vert\) seeding。% remark: verify argument on $\epsilon$-Coreset

{\fontsize{10}{12}\selectfont

	\begin{longtable}{llll}
		% \small \\
		\caption{理论保证及加速相关算法结果汇总}
		\label{tab: kmeans_algorithms} \\
		\toprule
		算法 & 采样量 & 时间复杂度 & 聚类质量 \\
		\midrule
		\multirow{2}{*}{$k$-means++} & \multirow{2}{*}{$k$} & \multirow{2}{*}{$O(nkd)$} & $\E[\varphi_C(\mathcal{X})] \leq 8(\ln k + 2)$\\
		& & & $\cdot \varphi_{\text{OPT}}(\mathcal{X})$ \\
		% remark: 18+\epsilon n^3 \epsilon^{-d}
		p-swap & $k$ & $O(n^3)$ & $\varphi_C(\mathcal{X}) \leq 18 \varphi_{\text{OPT}}(\mathcal{X})$ \\
		
		$k$-means++ & \multirow{2}{*}{$k$} & \multirow{2}{*}{$O(ndk^2 \log\log k)$} & \multirow{2}{*}{$\E[\varphi_C(\mathcal{X})] \in O(\varphi_{\text{OPT}}(\mathcal{X}))$}\\
		\&swap & & & \\

		$k$-means\(\vert \vert\) & \multirow{2}{*}{$O(kt)$} & \multirow{2}{*}{$O(ndkt)$} & $\E[\varphi_S(\mathcal{X})] \leq (\frac{1+\alpha}{2})^t \xi$\\
		sampling(1) & & & $+\frac{16}{1-\alpha} \varphi_{\text{OPT}}(\mathcal{X})$ \\
		
		$k$-means\(\vert \vert\) & \multirow{2}{*}{$O(lt)$} & \multirow{2}{*}{$O(ndlt)$} & $\E[\varphi_S(\mathcal{X})] \leq 2(\frac{k}{el})^t \text{Var}(\mathcal{X})$ \\
		sampling(2) & & & $+ 26\varphi_{\text{OPT}}(\mathcal{X})$\\

		$k$-means\(\vert \vert\)& \multirow{2}{*}{$O(lt)$} & $O(ndlt)+O(ndlt)$ & $\E[\varphi_C(\mathcal{X})] \leq O(\alpha)$ \\
		seeding & & $+O(\mathcal{A}_c)$ & $\cdot \varphi_{\text{OPT}}(\mathcal{X})$ \\

		\multirow{2}{*}{K-M$\text{C}^2$} & \multirow{2}{*}{$k$} & \multirow{2}{*}{$O(k^3 d \log^2 n \log k)$} & $\E[\varphi_C(\mathcal{X})] \leq O(\log k)$  \\
		& & & $\cdot \varphi_{\text{OPT}}(\mathcal{X})$ \\

		\multirow{2}{*}{AFK-M$\text{C}^2$} & \multirow{2}{*}{$k$} & \multirow{2}{*}{$O(nd)+O(\frac{1}{\epsilon}k^2 d\log \frac{k}{\epsilon})$}& $\E[\varphi_C(\mathcal{X})]\leq 8(\log k+2)$ \\
		& & & $\cdot \varphi_{\text{OPT}}(\mathcal{X}) + \epsilon\text{Var}(\mathcal{X})$ \\

		\multirow{2}{*}{$\epsilon$-Coreset} & \multirow{2}{*}{$O(\frac{k\log k}{\epsilon^2} (dk\log k+\log\frac{1}{\delta}))$} & \multirow{2}{*}{$O(ndk\log\frac{1}{\delta})$} & $\E[\varphi_C(\mathcal{X})] \leq \alpha(1+O(\epsilon))$ \\
		& & & $\cdot \varphi_{\text{OPT}}(\mathcal{X})$ \\

		Lightweight & \multirow{2}{*}{$O(\frac{dk\log k + \log(1/\delta)}{\epsilon^2})$} & \multirow{2}{*}{$O(nd)+O(\mathcal{A}_c)$} & $\E[\varphi_C(\mathcal{X})] \leq \alpha \varphi_{\text{OPT}}(\mathcal{X})$ \\
		Coreset & & & $+ \alpha O(\epsilon) \text{Var}(\mathcal{X})$\\

		均匀不放回 & \multirow{2}{*}{$s = O(\ln \left(\frac{1}{\delta}\right) \frac{\Delta^{2} \alpha^{2}}{2 \beta^{2} m^{2}})$} & \multirow{2}{*}{$O(s)+O(\mathcal{A}_c)$} & $\E[\varphi_C(\mathcal{X})] \leq (\alpha+\beta)$ \\
		采样(1) &  &  & $\cdot \varphi_{\text{OPT}}(\mathcal{X})$ \\

		均匀不放回 & \multirow{2}{*}{$s = O(\ln(\frac{1}{\delta})\frac{\alpha^2}{\beta^2}k^2\log^4 n)$} & \multirow{2}{*}{$O(k^2 \log^4 n)+O(\mathcal{A}_c)$} & $\E[\varphi_C(\mathcal{X})] \leq (\alpha+\beta)$ \\
		采样(2) &  &  & $\cdot \varphi_{\text{OPT}}(\mathcal{X})$ \\

		\multirow{2}{*}{算法\ref{alg: reduce_k-means2}(1)} & \multirow{2}{*}{$O(k)$} & $O(nkd)+O(nkd)$ & $\E[\varphi_C(\mathcal{X})] \leq O(\alpha)$ \\
		& & $+O(\mathcal{A}_c)$ & $\cdot \varphi_{\text{OPT}}(\mathcal{X})$ \\

		\multirow{2}{*}{算法\ref{alg: reduce_k-means2}(2)} & \multirow{2}{*}{$s = \Theta\left(\frac{k \log n}{\delta^{d / 2} \varepsilon^{d}} \log ^{d / 2}\left(\frac{k \log n}{\delta^{d / 2} \varepsilon^{d}}\right)\right)$} & $O(nsd)+O(nsd)$ & $\E[\varphi_C(\mathcal{X})] \leq \alpha(1+O(\epsilon))$ \\
		& & $+O(\mathcal{A}_c)$ & $\cdot \varphi_{\text{OPT}}(\mathcal{X})$ \\

		\bottomrule
	\end{longtable}

}

梳理完算法后,我们从理论上证明了基于均匀采样的算法\ref{alg: uniform_k-means}能有更好的近似系数,具体来说从$4(\alpha+\beta)$降到了$\alpha+\beta$,并且在K-M$\text{C}^2$所依赖的两个假设,假设1和2成立的情况下,均匀采样的数目可以在$O(\log^4 n)$数量级,从而在次线性时间内取得了常数近似的解。
接着实验表明,通过多采样一些点,确实可以让均匀采样比K-M$\text{C}^2$有更好的聚类质量,在传统聚类任务和图像分割任务中这一点均有体现。
最后我们指出一些可能的未来研究方向:
\begin{enumerate}[label=\arabic*)]
	\item 在次线性时间内有理论保证且无需假设的算法,比如近似系数是$O(\log n)$的等等
	\item 提出新的次线性时间算法,依靠新的易于验证的假设
	\item 探索基于距离的自适应采样的新性质,比如进一步缩小定理\ref{theo: k-means++ overseeding2}的界,比如定义$d'(x,C) = \norm{x - c_1}+\norm{x-c_2}$,$c_1$和$c_2$分别是$C$中离$x$最近的和第二近的点,根据$d'$来挑选类,增加跳转到不同最优类的概率,以改进理论界等等。
	\item 估算权重以加速算法\ref{alg: reduce_k-means2}
	\item 考察$k$-means\(\vert \vert\) sampling是否可以形成Coreset,从而改进$k$-means\(\vert \vert\) seeding的理论界。
\end{enumerate}
随着新的分析技术的发展,$k$-means问题历久弥新,新的
理论结果依然在不断出现,实用的算法层出不穷,随着大数据
时代的到来,该问题必将焕发出新的活力。
