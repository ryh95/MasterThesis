\section{基于均匀采样算法的新理论结果}
既然可以把谱聚类转换成带权kernel $k$-means问题,那么前一章的算法\ref{alg: uniform_k-means}是不是可以扩展到谱聚类上呢?答案是显然的,扩展后的算法如下,
\begin{algorithm}
    \caption{基于均匀不放回采样的带权kernel $k$-means算法(直觉)}
    \KwIn{映射后的数据集$\phi(\mathcal{X})$,类数目$k$,采样数$s$}
    \KwOut{$\phi(\mathcal{X})$的$k$个划分}
    $S \gets$ 从$\phi(\mathcal{X})$中均匀采样$s$个点\\
    $C \gets$ 在带权$S$上运行一个$\alpha$近似算法\\
    把$\phi(\mathcal{X})$靠到离自己最近的$C$上,得到数据点的$k$个划分\\
    \textbf{返回}$\phi(\mathcal{X})$的$k$个划分
\end{algorithm}
但是这样的算法没有办法实现,因为一般情况下我们拿不到点经过映射后的新特征,有的$\phi$会把点映射到无穷维,我们能参与运算的只有kernel,所以真正的算法见算法\ref{alg: uniform_wkk},这里$W(S,S)$和$K(S,S)$是基于采样索引获得的$W$和$K$的子矩阵,$Y\in B^{n\times k}$,该算法时间复杂度是$O(s^2+ns)$(如果$D$是单位阵的话)。类似定理\ref{theo: uniform_k-means},文献\cite{Mohan:2017:BNA:3172077.3172235}得出了带权kernel $k$-means下的对应定理。
\begin{theorem}[均匀不放回采样的谱聚类解的质量]
    \label{theo: uniform_wkk}
    令 $0 < \delta <1/2$, $\alpha \geq 1$, $\beta >0$是近似的参数。令$C$是由算法\ref{alg: uniform_wkk}中$\pi$所对应的中心点. 假设我们均匀不放回的采样$s$个点,如果,
    \begin{equation*}
    s \geq \ln(\frac{1}{\delta})(1+\frac{1}{n})/(\frac{\beta^2 m^2}{2\Delta^2 \alpha^2}+\frac{\ln(1/\delta)}{n})
    \end{equation*}
    则,我们有
    \begin{equation*}
    Ncut \leq 4(\alpha + \beta) Ncut^* + 4(\alpha + \beta)c
    \end{equation*}
    以至少 $1-2\delta$的概率, 其中$\Delta = \max\limits_{i,j}\norm{\phi(x_i) - \phi(x_j)}^2$是高维下数据直径的平方, $m = \Psi(\mathcal{X},C_{\text{OPT}_k}^{\mathcal{X}})/n$是点到其对应的中心点的距离的平均值,$c$是一个划分无关的常数,$Ncut^*$是最优的Ncut。
\end{theorem}
\begin{algorithm}
	\SetNoFillComment
    \caption{基于均匀采样和带权kernel $k$-means的谱聚类算法}\label{alg: uniform_wkk}
    \KwIn{数据集$\mathcal{X}$,类数目$k$,采样数$s$,相似度矩阵$A$,度矩阵$D$}
    \KwOut{$\mathcal{X}$的$k$个划分}
    $S \gets$ 从$1,2,...,n$中均匀采样$s$个数字\\
    $W \gets D,K \gets D^{-1}AD^{-1}$\\
    $W_s \gets W(S,S), K_s \gets K(S,S)$\\
    $\pi \gets$ 基于$W_s$和$K_s$,在带权$S$上运行一个$\alpha$近似算法\\
    \tcc{把点靠到高维中由采样点得到的中心点上}
    \For{$i = 1,...,n$}{
    	\For{$c = 1,...,k$}{
    		\begin{equation*}
	    	d(a_i,m_c) \gets K_{i i}-\frac{2 \sum_{a_{j} \in \pi_{c}} w_{j} K_{i j}}{\sum_{a_{j} \in \pi_{c}} w_{j}}+\frac{\sum_{a_{j}, a_{l} \in \pi_{c}} w_{j} w_{l} K_{j l}}{\left(\sum_{a_{j} \in \pi_{c}} w_{j}\right)^{2}}
	    	\end{equation*}\\
    	}
    	$j \gets \argmin\limits_{c = 1,2,...,k} d(a_i,m_c)$\\
    	$Y_{ij} \gets 1$
    }
    \textbf{返回}$Y$
\end{algorithm}
类似前面我们的更紧理论界,定理\ref{theo: uniform_k-means_sharper},这里也可以把定理\ref{theo: uniform_wkk}做的更紧,从而得出下面的定理。
\begin{theorem}[均匀不放回采样的谱聚类解的质量]
    \label{theo: uniform_wkk_sharper}
    令 $0 < \delta <1/2$, $\alpha \geq 1$, $\beta >0$是近似的参数。令$C$是由算法\ref{alg: uniform_wkk}中$\pi$所对应的中心点. 假设我们均匀不放回的采样$s$个点,如果,
    \begin{equation*}
    s \geq \ln(\frac{1}{\delta})(1+\frac{1}{n})/(\frac{\beta^2 m^2}{2\Delta^2 \alpha^2}+\frac{\ln(1/\delta)}{n})
    \end{equation*}
    则,我们有
    \begin{equation*}
    Ncut \leq (\alpha + \beta) Ncut^* + (\alpha + \beta)c
    \end{equation*}
    以至少 $1-2\delta$的概率, 其中$\Delta = \max\limits_{i,j}\norm{\phi(x_i) - \phi(x_j)}^2$是高维下数据直径的平方, $m = \Psi(\mathcal{X},C_{\text{OPT}_k}^{\mathcal{X}})/n$是点到其对应的中心点的距离的平均值,$c$是一个划分无关的常数,$Ncut^*$是最优的Ncut。
\end{theorem}
由于证明和前面$k$-means的很相似,这里不再详细展开,唯一需要注意的是在得到$\Psi(\mathcal{X},C) \leq (\alpha+\beta) \Psi(\mathcal{X},C_{\text{OPT}_k}^{\mathcal{X}})$后,将式\ref{eq: wkk_spectral_clustering}带入即可完成证明。