\chapter{\texorpdfstring{$k$}{k}-means聚类}
本章聚焦$k$-means问题,前面已经提到解决该问题的一个经典算法Lloyd算法,由于文献\cite{garey1982complexity,kleinberg1998microeconomic,mahajan2009planar}指出$k$-means问题是NP难的,所以该算法准确来说是一个近似算法。前面提到,该算法收到初始点影响大,所以本章将首先探讨有理论保证的$k$-means问题的算法。其次,$O(nkdt)$的时间复杂度在海量数据下也难以处理,所以接着我们会探讨如何加速。然后,我们探讨如何将这两点都覆盖,最后我们给出我们的贡献并总结本章。

\section{\texorpdfstring{$k$}{k}-means问题引入及背景知识}
我们首先定义$k$-means问题,给定点$x \in \R^d$和集合$C \subseteq \R^d$,$d$是数据的维度,我们可以定义该点到集合的距离
\begin{equation*}
d(x,C) = \min\limits_{c \in C}\norm{x - c}
\end{equation*}
给定$n$个点的数据集$\mathcal{X} \subseteq \R^d$和集合$C \subseteq \R^d$,我们可以定义如下目标函数
\begin{equation*}
\varphi(\mathcal{X},C) = \sum_{x \in \mathcal{X}}d^2 (x,C)
\end{equation*}
我们的目标是在给定$n$个点$\mathcal{X}$情况下找到大小为$k$的集合$C_{\text{OPT}_k}^\mathcal{X}$使得该目标函数最小,即
\begin{equation*}
C_{\text{OPT}_k}^\mathcal{X} = \argmin\limits_{ \substack{C \subseteq \R^d \\ |C| = k} }\varphi(\mathcal{X},C)
\end{equation*}
上文谈到解的质量,这里给出如下定义用于定量评价解的质量。
\begin{definition}[$k$-means问题解的质量]
    \label{def: k-means_quality}
    一个算法 $\mathcal{A}$被称为$k$-means问题的一个$\alpha$近似算法如果存在一个数字$\alpha \geq 1$使得
    \begin{equation*}
        \varphi(\mathcal{X},C) \leq \alpha \varphi(\mathcal{X},C_{\text{OPT}_k}^{\mathcal{X}})
    \end{equation*}
    成立,此时称解$C$为
一个$\alpha$近似解,这里$\alpha$称为近似系数。
\end{definition}
从该定义可以看出, $\alpha$越小解质量越好,越接近最优解。