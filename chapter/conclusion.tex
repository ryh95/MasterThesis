\chapter{全文总结与展望}

\section{全文总结}
本文研究了大数据下扁平聚类的两个问题,分别是如何加速聚类以及如何获得更好的解的。第一章,我们引入了聚类的概念,建立了聚类的工作流,简单介绍了常用的聚类算法,随后指出了本文的研究目标和行文结构。在本文中,我们具体研究了$k$-means问题和谱聚类问题,第二章,在$k$-means问题中,我们介绍了基于$k$-means++和局部搜索的方法来改进Lloyd算法的聚类质量,用各种采样方法(均匀不放回、Coreset)来减少数据量从而加速了$k$-means的求解,我们将采样和理论保证的方法拼在一起让$k$-means可以又快又好的被解决。接着,我们给出了我们的贡献,首先,$k$-means++算法被扩展到了带权重的$k$-means问题上,我们证明了这一扩展后的方法可以取得$O(\log k)$的近似,其次,我们证明了基于均匀不放回采样方法可以有更紧的$k$-means理论界,同时,在温和的数据假设下,均匀采样的数目可以被限定在$O(\log^4 n)$,从而让$k$-means可以在多项式对数时间复杂度下获得常数近似的解,在本章的最后,通过和K-M$\text{C}^2$以及新提出的Double-K-M$\text{C}^2$做实验上的对比从而实践中验证了基于均匀采样的方法可以加速$k$-means的求解且获得有保证的聚类质量。

然后,在第三章中,我们探索了谱聚类问题,谱聚类能够解决$k$-means不能解决的非线性划分的问题,在实践中效果好。然而,这一算法需要谱分解,用常规的分解方法需要$O(n^3)$的时间复杂度,所以我们首先介绍了如何加速谱聚类。加速的方法分为三种,第一种是用Nyström方法低秩近似,这一方法的副产物是近似的特征向量,由于只需要分解采样的小矩阵,时间复杂度大大减少,第二种方法是减少数据量,这一方法更为直接,即在采样点上聚类,然后利用数据点和采样点的关系得到所有点的聚类,最后一种方法利用采样构建相似度矩阵,这一矩阵本身就是低秩的,分解的是$n\times m$的矩阵,时间复杂度从$O(n^3)$降到了$O(nm^2)$,$m$是采样数。随后,我们指出了改进谱聚类的方法,一种是用谱旋转改进谱分解后的聚类,另一种方法是将谱聚类问题转换为带权kernel $k$-means问题,再用前述的理论保证算法给谱聚类提供保证。随后,我们给出了本文在谱聚类上的贡献,将均匀不放回方法扩展到带权kernel $k$-means上,证明了谱聚类的更紧的理论界,随后,在实验环节,我们在实践中确认了通过均匀采样,谱聚类速度可以很快,同时聚类质量又不会损失很多。

\section{后续工作展望}
在提出$k$-means问题和Lloyd算法的50、60年后,聚类依然在蓬勃发展。考虑到连贯性,在2、3章的最后,我们分别给出了一些可能的$k$-means和谱聚类的未来可能方向,这里我们站在一个更高的角度,尝试回答对于聚类来说,未来的研究方向和可能的重要的未解决问题,我们认为有以下问题值得探索。

聚类算法的选择。在这个问题上,传统的理论派和实践派都忽视了,理论派往往关心如何加速聚类,如何证明更好的理论结果等等,而对于实践派来说,聚类算法的选择往往过于随意,比如速度快、程序开源、算法好懂等,但是这些选择方式忽略了本质,即,我们应该根据实践中问题的需要选择算法。对于聚类来说,需求可能会有很多,比如相似的点要在一个类,不相似的在不同的类,不同的类的大小要平衡,不能有的类很大,有的类很小等。需要注意的是,这些要求往往不能同时满足,比如文献\cite{ben2018clustering}指出相似的点要在一个类,不相似的在不同的类这两个要求有的时候会互相冲突,所以我们需要明确在自己的任务中哪些聚类要求是主要的,并且,我们需要知道不同的算法会满足哪些需求,即算法的偏好(bias),我们应该根据这两个点来选择我们的算法,不过对于聚类算法的终端用户来说,他们不是聚类算法研究人员,他们往往很难用术语说清自己的聚类需要什么需求。所以,对于聚类算法的选择依然是一个开放性问题。

弥补实践和理论的鸿沟。对于传统的理论派来说,他们往往分析的是最差情况下的算法理论界,对于$k$-means和谱聚类这样的NP难问题,最差情况下算法的时间复杂度是指数级。但是,在现实情况中,简单的经验性算法,比如Lloyd算法一般也能取得不错的效果,另一方面,对于一个3层的中间层神经元数目超过3的神经网络来说,训练它都是NP难的问题,但是现实却是我们取得了新一轮的深度学习的辉煌,这种现实和理论脱离的感觉不免让人疑惑。这中间缺失了什么使得现实的情况比理论情况好的多?在聚类上,或许是现实的数据更加“可分”?事实上,有一些工作通过添加数据的假设说明了在这些数据上聚类是容易的,算法可以在多项式时间结束。不过,多数假设非常难验证,而且,通过仔细分析,可以发现这些假设对数据施加的限制很多是不现实的,所以对于理论派来说,未来需要提出更好的能够易于验证的假设,且需要增加自己的理论工具来放松对数据的限制从而解释现实情况,更进一步的,理论派需要在解释现实情况的基础上去指导未来算法的设计。

其他问题。这里提一些潜在的小问题,比如类数目$k$怎么选?如果$k$在一个范围内都是合适的,怎么设计聚类的目标函数?聚类的理论基石等等。

最后,随着新硬件的不断加持和理论工具的逐渐扩充,我们有理由相信未来以聚类为代表的无监督方法能够引领人工智能的前进,能够更好的处理海量数据,从而让未来人类在技术的帮助下享受美好的人生。
