\section{加速谱聚类}
谱聚类虽然效果好,能够解决$k$-means不能解决的非线性可分情况,但是其谱分解过程传统方法需要$O(n^3)$的时间复杂度,这对于大数据来说是不能够忍受的,本节,我们将从低秩近似相似度矩阵、减少数据量、低秩构建相似度矩阵三方面加速谱分解。
\subsection{低秩近似相似度矩阵}
Nyström 方法是一种半正定矩阵近似方法,可以有效近似半正定矩阵,该方法由文献\cite{williams2001using}引入到机器学习领域。其核心思想是:通过在原矩阵中采样部分列或者行,用这部分信息对原矩阵进行低秩近似。它也可以近似特征值和特征向量,通过分解大矩阵中小的子矩阵来近似大矩阵的特征值和特征向量,方法详解如下。
假设我们有一个半正定矩阵$K \in \R^{n \times n}$,因为它是半正定,所以有
\begin{equation*}
    K = X^T X
\end{equation*}
可将$X$分块,$X=[R\quad S]$,$X \in \R^{d\times n}$,其中$R \in \R^{d\times m}$,所以
\begin{align*}
    K & = \begin{bmatrix}
            R^T R & R^T S \\
            S^T R & S^T S 
          \end{bmatrix} 
\end{align*}
另一方面,我们也可以通过某种采样方法(比如均匀采样)选择$K$的$m$行(或者列),一般来说$K$表示的是数据点之间的关系,所以可以调整选中行或者列在$K$中的顺序而不会影响原始信息,不妨将选中的行调整到$K$的前排,没被选中的都放到后排,从而将$K$分块表示为如下形式
\begin{align*}
    K & = \begin{bmatrix}
            A & B \\
            B^T & C
          \end{bmatrix} 
\end{align*}
其中$[A \quad B]$是我们采样的行,此时,我们令$A = R^T R$,$B = R^T S$(通常情况,等式不会自然成立,这里我们令他们成立,目的是用$A,B$来表示$C$)。首先,我们对$A$做谱分解求得$R$,
\begin{equation*}
    A = R^T R = U\Sigma U^T
\end{equation*}
即,$R = \Sigma^{\frac{1}{2}}U^T$,又因为
\begin{equation*}
    B = R^T S = U\Sigma^{\frac{1}{2}}S
\end{equation*}
所以,$S = \Sigma^{-\frac{1}{2}}U^T B$,
此时$C \approx B^T A^{-1} B$,记用$A,B$近似的$K$为$\hat{K}$,容易知道它是$K$的一个低秩近似,秩最多是$m$,将$B^T A^{-1} B$带入$\hat{K}$有
\begin{align*}
    \hat{K} & = \begin{bmatrix}
            A & B \\
            B^T & B^T A^{-1} B
          \end{bmatrix} 
          = \begin{bmatrix}
              U\Sigma U^T & B \\
            B^T & B^T U \Sigma^{-1} U^T B
          \end{bmatrix}
          = \begin{bmatrix}
            U \\
            B^T U \Sigma^{-1}
          \end{bmatrix} \Sigma
          \begin{bmatrix}
            U^T & \Sigma^{-1}U^T B
          \end{bmatrix}
          = K_{:1}A^{-1}K_{:1}^T
\end{align*}
也就是说,我们可以用$\begin{bmatrix} U \\ B^T U \Sigma^{-1} \end{bmatrix} = K_{:1}U\Sigma^{-1}$来近似$K$的特征向量,用$A$的特征值$\Sigma$来近似$K$的特征值,这里$K_{:1} = [A \quad B]^T$是我们采样的行。由于只需要对$A$这个$\R^{m\times m}$的矩阵做分解就可以近似大矩阵的特征值和特征向量,故我们加速了大矩阵的分解。完整Nyström算法见\ref{alg: Nyström method},该算法时间复杂度是$O(nm^2 + m^3)$。
\begin{algorithm}
    \caption{Nyström 方法}\label{alg: Nyström method}
    \KwIn{半正定矩阵$K$,采样数目$m$}
    \KwOut{低秩矩阵$\hat{K}$}
    $S \gets$ 根据某些采样算法得到的列(或者行)的下标 \\
    $A \gets K(S,S)$ \\
    $K_{:1} \gets K(:,S)$ \\
    令$A = U\Sigma U^T$是$A$的谱分解,$U,\Sigma$分别是$A$的特征向量和特征值。\\
    \textbf{返回} 低秩矩阵$\hat{K} = K_{:1}A^{-1}K_{:1}^T$或者近似的特征向量$\hat{U}_K = K_{:1}U\Sigma^{-1}$
\end{algorithm} 
% 那求出$R$和$S$和加速特征值分解有什么关系呢?注意到我们实际要得到的特征向量可以从$X$处轻松获得,而要获得$X$,我们只需要对$A$这个$\R^{m\times m}$的矩阵做分解就可以获得,故我们加速了大矩阵的分解。
对于Nyström 方法,我们应该注意什么呢?第一,矩阵$K$应该本身就是低秩的,否则使用Nyström 方法将没有意义。第二,行的选择很关键,如果我们选到了大量的线性相关的行,则估计误差就会很大。
\begin{theorem}
    \label{theo: nystrom_uniform}
    令$\epsilon \in (0,1)$,$\delta \in (0,1)$,考虑用均匀采样$m$列(可以放回也可以不放回),则当$m$满足$m \geq 2 \epsilon^{-2} \mu k \log (k / \delta)$时
    \begin{equation*}
        \|K-\hat{K}\|_{2} \leq\left(1+\frac{n}{(1-\varepsilon) m}\right)\left\|K-K_{k}\right\|_{2}
    \end{equation*}
    以至少$1-3\delta$的概率成立,$\norm{.}_2$是矩阵的谱范数(spectral norm),其中
    \begin{equation*}
        \mu=\frac{n}{k} \max _{i=1, \ldots, n}\left\|V_{k}(i,:)\right\|_{2}^{2}
    \end{equation*}
    记$V$是将$K$的特征值从大到小排序后对应的特征向量,$V_{k}$是$V$的前$k$列,$K_k$是$K$的最优的秩为$k$的近似矩阵。
\end{theorem}
常用的的采样(列选择)是均匀采样,基于文献\cite{gittens2016revisiting}的结论,定理\ref{theo: nystrom_uniform}给出了均匀采样的话Nyström方法的误差,这里$\mu$被称为$V_k$的coherence。可以看到如果coherence比较大的话,均匀采样效果不太好。因此,一些更加高级的采样方法被提了出来,比如以概率$\left\|V_{k}(i,:)\right\|_{2}^{2}/k$采样第$i$列,有放回的重复$m$次的采样方式,这个概率也被称为leverage score。
\begin{theorem}
    令$\epsilon \in (0,1)$,$\delta \in (0,1)$,考虑按照上面描述的高级采样方式采样$m$列,则当$m \geq O\left(\epsilon^{-2} k \log (k / \delta)\right)$时有
    \begin{equation*}
        \|K-\hat{K}\|_{2} \leq\left\|K-K_{k}\right\|_{2}+\epsilon^{2}\left\|K-K_{k}\right\|_{*}
    \end{equation*}
    以至少$0.8-2\delta$的概率成立,其中$\norm{.}_{*}$是矩阵的迹范数(trace norm)
\end{theorem}
当然,精确计算leverage score需要对$K$做SVD,而这正是我们想避免的。尽管如此,如果$K$有$O(e)$个非零元素的话,可以在差不多$O(ek \log e)$的时间复杂度下对leverage score做近似,这里近似的误差是一个乘性误差。
到目前为止最快的利用leverage score的Nyström方法的时间复杂度接近$n$的线性函数,该算法用了一种复杂的递归的采样方式\citing{musco2017recursive}。

那么怎么将Nyström方法用到谱聚类上呢?令$A \in \R^{n \times n}$是谱聚类的相似度矩阵,它对应的归一化的拉普拉斯矩阵是$L_n = I - D^{-1/2}AD^{-1/2}$,回忆求解谱聚类需要找到$L_n$的最小的$k$个特征值对应的特征向量,这等价于找到$M = D^{-1/2}AD^{-1/2}$最大的$k$个特征值对应的特征向量,所以我们的目的就是要近似这些特征向量,通过前面的分析,我们可以通过对$A$分块拿到$A$的近似特征向量,这里的基本思想是用$A$的分块来表示$M$的分块,然后用$M$的分块利用Nyström方法近似$M$的特征向量,该方法由文献\cite{fowlkes2004spectral}提出,详解如下。我们将$A$做如下分块
\begin{equation*}
    A = \left[\begin{array}{c|c}
        A_{11} & A_{21}^T \\
      \hline
      A_{21} & A_{22}
    \end{array}\right]
\end{equation*}
我们知道的是$A_{11}$和$A_{21}$,根据Nyström方法我们可以得到一个低秩近似$\hat{A}$,根据$\hat{A}$可以得到一个近似的度矩阵$\hat{D} = \text{diag}(\hat{A}\mathbf{1})$,$\text{diag}(v)$是一个以向量$v$为对角元的对角矩阵,$\mathbf{1}$是一个元素全为1的向量,类比于$A$的分块,将$\hat{D}$分块。
\begin{equation*}
    \hat{A} = \left[\begin{array}{c|c}
        A_{11} & A_{21}^T \\
      \hline
      A_{21} & A_{21}A_{11}^{-1}A_{21}^T
    \end{array}\right],
    \hat{D} = \left[\begin{array}{c|c}
        D_{11} &  \\
      \hline
      & \hat{D}_{22}
    \end{array}\right]
\end{equation*}
有了$\hat{A}$和$\hat{D}$可以近似$M$,令近似矩阵是$\hat{M}$,则有
\begin{equation*}
    \hat{M} = \hat{D}^{-1/2}\hat{A}\hat{D}^{-1/2} = \left[\begin{array}{c|c}
        M_{11} & \hat{M}_{21}^T \\
      \hline
      \hat{M}_{21} & \hat{M}_{22}
    \end{array}\right]
\end{equation*}
其中$M_{11} = D_{11}^{-1/2}A_{11}D_{11}^{-1/2}$,$\hat{M}_{21} = \hat{D}_{22}A_{21}D_{11}^{-1/2}$,记$\hat{M}_{:1} = \begin{bmatrix}
    M_{11} \\
    \hat{M}_{21}
\end{bmatrix}$,此时即可用算法\ref{alg: Nyström method}来近似$M$的特征向量,只不过此时算法\ref{alg: Nyström method}中的$A$是$M_{11}$,$K_{:1}$是$\hat{M}_{:1}$。

值得注意的是,这样求出来的的近似特征向量不可直接基于它用$k$-means聚类,原因是这个近似的特征向量列不正交,由于不正交,基于这样的特征向量可能得出的谱聚类的解质量很差,所以我们需要对现有近似特征向量做后处理,使其正交。在文献\cite{fowlkes2004spectral}中,作者根据分块矩阵$M_{11}$是否是半正定的提出了两种方法。

(一)$M_{11}$是半正定

首先,考虑将$\hat{M}$写成谱分解的样子
\begin{align*}
    \hat{M} &= \begin{bmatrix}
        M_{11} \\ \hat{M}_{21}
    \end{bmatrix} M_{11}^{-1}
    \begin{bmatrix}
        M_{11} & \hat{M}_{21}^T
    \end{bmatrix} \\
    &= \left( \begin{bmatrix}
        M_{11} \\ \hat{M}_{21}
    \end{bmatrix} M_{11}^{-1/2} U \Sigma^{-1/2} \right) \Sigma \left( \Sigma^{-1/2} U^T M_{11}^{-1/2} \begin{bmatrix}
        M_{11} & \hat{M}_{21}^T \end{bmatrix} \right) \\
    &= V\Sigma V^T
\end{align*}
上式中$\Sigma$是任意对角矩阵,$U$是任意正交矩阵,为了求出$U$和$\Sigma$,我们要求$V$列正交,有
\begin{align*}
    I &= V^T V \\
    &= \left( \Sigma^{-1/2} U^T M_{11}^{-1/2} \begin{bmatrix}
        M_{11} & \hat{M}_{21}^T \end{bmatrix} \right) \left( \begin{bmatrix}
        M_{11} \\ \hat{M}_{21} \end{bmatrix} M_{11}^{-1/2} U \Sigma^{-1/2} \right) \\
\end{align*}
对上面等式左乘$U \Sigma^{1/2}$,右乘$\Sigma^{1/2} U^T$,有
\begin{align*}
    U\Sigma U^T &= \left( M_{11}^{-1/2} \begin{bmatrix}
        M_{11} & \hat{M}_{21}^T \end{bmatrix} \right) \left( \begin{bmatrix}
        M_{11} \\ \hat{M}_{21} \end{bmatrix} M_{11}^{-1/2} \right) \\
        &= M_{11} + M_{11}^{-1/2} \hat{M}_{21}^T \hat{M}_{21} M_{11}^{-1/2}
\end{align*}
所以说,我们只要对$S = M_{11} + M_{11}^{-1/2} \hat{M}_{21}^T \hat{M}_{21} M_{11}^{-1/2}$做谱分解,得到它的特征向量$U_S$和特征值$\Sigma_S$就可以得到$M$的近似特征向量$V = \begin{bmatrix} M_{11} \\ \hat{M}_{21} \end{bmatrix} M_{11}^{-1/2} U_S \Sigma_S^{-1/2}$

(二)$M_{11}$不是半正定

如果$M_{11}$不是半正定,$M_{11}^{-1/2}$就没有定义,前面方法就不可以正交化,文献\cite{fowlkes2004spectral}中将情况1的方式称为one-shot,因为不用求出不正交的特征向量,而这里需要按照以下两步进行。首先,根据前面的方法得出近似的但是不正交的特征向量$\bar{V} = \hat{M}_{:1}U_{M_{11}}\Sigma_{M_{11}}^{-1}$,其中$U_{M_{11}}$和$\Sigma_{M_{11}}^{-1}$分别是$M_{11}$的特征向量和特征值。接着,取$Z = \bar{V}\Sigma_{M_{11}}^{1/2}$,这样$\hat{M} = ZZ^T$,对$Z$做奇异值分解,$U_Z$即为所求的正交化处理后的特征向量
\begin{equation*}
    Z = U_{Z}\Sigma_{Z}V_Z^T
\end{equation*}
这个基于Nyström方法加速谱聚类的方法总结见算法\ref{alg: Nyström spectral},这里$\hat{U}_{:k}$是$\hat{U}$最大的$k$个特征值对应的特征向量。该算法的时间复杂度是$O(nm^2+m^3)$,如果$m$比较大,时间开销还是会比较大,文献\cite{li2011time,choromanska2013fast}对此做了进一步改进。
% remark: 调整算法的输入,让算法更清楚,写算法的时间复杂度
\begin{algorithm}
    \caption{Nyström 方法加速谱聚类}\label{alg: Nyström spectral}
    \KwIn{近似的相似度矩阵$\hat{A}$}
    \KwOut{归一化拉普拉斯矩阵$L_n$最小的$k$个特征值对应的特征向量}
    $\hat{D} \gets \text{diag}(\hat{A}\mathbf{1})$\\
    $M_{11} \gets D_{11}^{-1/2}A_{11}D_{11}^{-1/2}$ \\
    $\hat{M}_{21} \gets \hat{D}_{22}A_{21}D_{11}^{-1/2}$ \\
    \uIf{$M_{11}$是半正定矩阵}{
        $S = M_{11} + M_{11}^{-1/2} \hat{M}_{21}^T \hat{M}_{21} M_{11}^{-1/2}$ \\
        $U_S,\Sigma_S \gets $ 对$S$做谱分解得到它的特征向量和特征值 \\
        $\hat{U} = \begin{bmatrix} M_{11} \\ \hat{M}_{21} \end{bmatrix} M_{11}^{-1/2} U_S \Sigma_S^{-1/2}$
    }
    \Else{
        $U_{M_{11}},\Sigma_{M_{11}} \gets $ 对$M_{11}$做谱分解得到它的特征向量和特征值 \\
        $Z \gets \begin{bmatrix} M_{11} \\ \hat{M}_{21} \end{bmatrix} U_{M_{11}} \Sigma_{M_{11}}^{-1/2}$ \\
        $\hat{U} \gets $ 对$Z$做奇异值分解得到它的左奇异矩阵
    }
    \textbf{返回} $\hat{U}_{:k}$
\end{algorithm}

\subsection{减少数据量}
类似于$k$-means问题,谱聚类中一种自然的加速方法也是减少数据量,这个想法的框架如下:
\begin{enumerate}
    \item 采样$m$个数据点,将这些点称为采样集
    \item 在采样集上运行谱聚类算法
    \item 再将剩下的点靠到已有的解上得到全部数据点的划分
\end{enumerate}
这个框架下一个经典的算法是KASP\citing{yan2009fast},算法描述见算法\ref{alg: KASP},
\begin{algorithm}
    \caption{KASP}\label{alg: KASP}
    \KwIn{数据集$\mathcal{X} = \{x_1,x_2,...,x_n\}$,类数目$k$,采样数目$m$}
    \KwOut{$\mathcal{X}$的$k$个划分}
    在$\mathcal{X}$上运行Lloyd算法,令$k=m$,得到中心点$Y = \{y_1,y_2,...,y_m\}$\\
    对$\mathcal{X}$和$Y$建立关联字典,$x_i:y_j$(key:value),其中$y_j$是离$x_i$最近的$Y$中的点 \\
    在$Y$上做一个谱聚类,类数目是$k$ \\
    对任意$x_i$,根据$\mathcal{X}$和$Y$的关联字典查找$y_j$,再根据$y_j$的类id,得到$x_i$的类id\\
    \textbf{返回}$\mathcal{X}$的$k$个划分
\end{algorithm}
算法时间复杂度是$O(nmdt+m^3)$,其中$t$是Lloyd的迭代次数。那么,这个算法的近似误差是多少呢?这里,我们对文献\cite{yan2009fast}的分析做一个简单的总结。在算法\ref{alg: KASP}中$x_i$到$y_j$的关系是通过$k$-means建立的,这个$x_i$到$y_j$的关系也可以看做是$x_i$经过扰动得到的$y_j$,即$x_i+\epsilon_i = y_j$,这里$\epsilon_i$是数据点的扰动,所以这里的$k$-means可以看做一种将$\mathcal{X}$扰动到$Y$的过程,扰动后的结果是$$\mathcal{X}' = \{y_1,...,y_1,y_2,...,y_2,...,y_k,...,y_k\}$$,令$y_i$的重复次数是$r_i$,则$\sum_{i=1}^k r_i = n$。一般的,我们令扰动后的结果是$\tilde{\mathcal{X}} = \{x_1+\epsilon_1,...,x_n+\epsilon_n\}$,考虑基于$\mathcal{X}$和$\tilde{\mathcal{X}}$的谱聚类结果的区别$\rho$,定义$$\rho=\frac{1}{n} \sum_{i=1}^{n} \mathbb{I}\left\{I_{i} \neq \tilde{I}_{i}\right\}$$,$\mathbb{I}$是指示函数,内里表达式如果为真就是1,$I = (I_1,...,I_n)$,$I_i$表示了点$x_i$的类id,$\tilde{I} = (\tilde{I}_1,...,\tilde{I}_n)$,$\tilde{I}_i$表示了点$\tilde{x}_i$的类id。基于文献\cite{huang2009spectral}的工作,可得如下定理。
\begin{theorem}[KASP的理论保证]
    在文献\cite{yan2009fast}中定理3假设条件成立情况下,有$\rho \leq O\left(\frac{k}{g_{0}^{2}}\|L-\tilde{L}\|_{F}\right)$,其中$g_0$是一个依赖谱间距(spectral gap)的值,$L$和$\tilde{L}$分别对应$\mathcal{X}$和$\tilde{\mathcal{X}}$的拉普拉斯矩阵。同时,如果文献\cite{yan2009fast}中定理6假设成立,则下式会以一个大概率成立
    \begin{equation*}
        \|L-\tilde{L}\|_{F} \leq O\left(\sigma_{\epsilon}^{(2)}+\sigma_{\epsilon}^{(4)}\right)
    \end{equation*}
    其中$\sigma_{\epsilon}^{(2)}$和$\sigma_{\epsilon}^{(4)}$是扰动项$\epsilon$的范数$\norm{\epsilon}$的二阶和四阶矩。
\end{theorem}
由上面两个理论界,$\rho$就可以被$\norm{\epsilon}$的二阶和四阶矩限定住,而Lloyd算法实际是在最小化二阶矩,从这一点来讲,用$k$-means得到采样集的过程有理论支持。除去Lloyd算法得到采样集,还有一个算法叫eSPEC也可以用来得到采样集,详见文献\cite{wang2009approximate}。

\subsection{低秩构建相似度矩阵}
除去上述的两种方法,还有一种基于landmark/anchor的方法\citing{liu2010large,cai2014large},landamark/anchor其实也是一些采样点,不过这里采样是为构建相似度矩阵做准备,这类方法思路如下。

首先,通过Lloyd或者一些其他的采样方法得到一个采样集$W \in \R^{d \times m}$,$m$是采样数,下一步是把这$m$个点当做一组基,用这组基去表示$n$个点使得$\mathcal{X} \approx WB$,这里$B \in \R^{n \times m}$即是根据基得到的表示矩阵。有了$B$,我们就可以用$B$去构建相似度矩阵$A$,$$A = B\Delta^{-1}B$$,这里$\Delta$是一个对角矩阵,其中$\Delta_{jj} = \sum_{i=1}^n B_{ij}$,这个对角阵起一个归一化的作用。这里用$BB^T$表示$A$的一个直觉是,如果两个点的$m$维表示比较接近的话,这两个点的相似度应该比较高,那么反过来可以用他们的内积来表示他们的相似度。这个相似度矩阵$A$的一个好处是,可以令$P = B\Delta^{-1/2}$,从而让$A = PP^T$,这样就可通过对$P$做奇异值分解得到$A$的特征向量,从而让分解的时间复杂度降到$O(nm^2)$。一种构建$B$的方法是求解下式\citing{chen2017scalable,nie2016constrained}。$$\min _{B_i \mathbf{1}=1, B_i \geq 0} \sum_{j=1}^{m} B_{i j}\left\|\mathbf{x}_{i}-\mathbf{w}_{j}\right\|_{2}^{2}+\gamma \sum_{j=1}^{m} B_{i j}^{2}$$,这里$B_i$是$B$的第$i$行。

综上,我们总结了三种基于采样的加速谱分解的方法,他们的共同点是都基于采样,但是各个方法的采样目的不同。对Nyström方法来说,采样是为了低秩近似相似度矩阵,而低秩近似带来的一个副产品是可以近似特征向量,这个方法是不用知道相似度矩阵是怎么构建的的,假设采样$m$个点,只需要计算$n \times m$的相似度矩阵,分解$m \times m$的那一部分就可以得到近似的特征向量,另一方面Nyström方法有很好的理论支撑,基于Nyström方法的大都有清晰的理论结果,Nyström方法的延伸也很广,可以广泛用在矩阵近似上,比如kernel近似等\citing{wang2014efficient,wang2014modified,sun2015review}。对KASP来说,采样是为了减少数据,分解的是一个$m \times m$的矩阵,缺点是得不到近似的特征向量,因此方法的扩展性不是很好。对基于landmark/anchor的方法来说,采样的目的是得到一个相似度矩阵,采样的副产品是这个相似度矩阵由于构建方法的关系可较为方便计算特征向量,这个方法的缺点也很明显,这个方法得到的特征向量和相似度构建方式紧密绑定,如果相似度矩阵是别人给的或者某种其他规定的方式生成的的话,这个方式是没有办法用的,另外,这个方法分解的时间复杂度也比前两种方法高。

由于都是采样方法,采样方式是很关键的,如果采样的列或者数据点线性相关性比较强的话,对于Nyström方法来说近似的矩阵的秩和真实的就会差很多,对基于landmark的方法来说$WB$的秩就会比真实数据小很多,从而损失很多信息。对于Nyström方法来说,除去均匀采样和利用leverage score的方法以外,还有很多其他的采样方式,文献\cite{kumar2012sampling}是一个好的总结。对于基于landmark/anchor的方法来说,常用的生成landmark的方法除了Lloyd还有比如BKHK\citing{zhu2017fast}、BKM\citing{chen2018spectral}等。最后,除了这里提到的三种采样方式外,最近的文献\cite{tremblay2020approximating}还提到了很多其他的方式并给出了有洞见的观点,是很好的参考资料。

% 都是采样,但是目的不同
% 其他anchor方法
% 不同方法缺点
% 共同点