\section{改进谱聚类聚类质量}
本节我们探究如何改进谱聚类的质量,我们从两个维度考虑,一个方面是谱分解后的聚类质量来提高谱聚类的质量,另一方面是,通过把谱聚类转换成带权kernel $k$-means问题,用前述的$k$-means的一些方法来给谱聚类提供理论保证。

\subsection{改进谱分解后的聚类}
谱分解后,$k$个$n$维的特征向量使得每个数据点变为了一个$k$维向量,这个过程也可以理解为每个数据点经过特征变换获得了新的特征,这些新的特征有时也会称为谱嵌入(spectral embedding),在传统的谱聚类中通过Lloyd算法将谱嵌入转化为离散的划分,但是Lloyd算法终究不够准确,这里我们考虑一些方法来改进这一转换的过程。文献\cite{stella2003multiclass}提出了一种后来被称为谱旋转\citing{huang2013spectral,chen2017scalable}的方法。回忆谱聚类的过程,原问题是式\ref{eq: ori_spectral_clustering},文献\cite{stella2003multiclass}对式\ref{eq: ori_spectral_clustering}做如下等价改写,
\begin{equation}
	\max _{Y \in I n d, Z=Y\left(Y^{T} D_{A} Y\right)^{-\frac{1}{2}}} \operatorname{Tr}\left(Z^{T} A Z\right)
\end{equation}
这里$Y \in B^{n \times k}$是0/1标记矩阵(目的是解出它),值只能取0/1,每一行只有一个1,$A$是相似度矩阵,$D_A$是$A$的度矩阵,对上式relax后得到下式
\begin{equation}
	\max _{Z^{T} D_{A} Z=I_{k}} \operatorname{Tr}\left(Z^{T} A Z\right)
\end{equation}
其中$I_k$是$k\times k$的单位阵,这个relaxation和式\ref{eq: relax_spectral_clustering}是等价的,这个问题的最优解$Z^*$是$D_A^{-1}A$最大的$k$个特征向量。注意到,如果对$Z^*$右乘正交矩阵,上面relax式目标值不会改变,换句话说,我们可以用$Z^*R$去逼近$Z$,这里的$R$是一个正交矩阵,这便是谱旋转。在文献\cite{stella2003multiclass}中,在求得$Z^*$后,作者对$Z^*$的每一行做了行归一化(每一行的行范数是1),得到$Y^*$。然后再用$Y^*R$去逼近$Y$,
\begin{equation}
 	\min_{Y \in I n d,R^TR = RR^T = I_k} \norm{Y - Y^*R}_F^2
\end{equation}
为什么要归一化呢?可能是作者觉得原来的$Y$每一行范数也是1,为了把relax后的解变回到原来的$Y$的空间,需要做行归一化。但是,注意到$Z$和$Y$的关系,$Z=Y\left(Y^{T} D_{A} Y\right)^{-\frac{1}{2}}$,也即是说$Y$是做了列的伸缩得到的$Z$,那么行的归一化就不是一种合理的操作。文献\cite{chen2017scalable}显然注意到了这一点,提出另一种更合理的谱旋转,核心思想是不把$Z^*$变为$Y^*$,直接让$Z^*R$去逼近$Z$,从而得到下式
\begin{equation}
	\min_{Y \in I n d,R^TR = RR^T = I_k} \norm{Y\left(Y^{T} D_{A} Y\right)^{-\frac{1}{2}} - Z^*R}_F^2
\end{equation}
对上式的求解得到了更好的聚类质量,那么为什么谱旋转会比Lloyd更适合呢?文献\cite{huang2013spectral}对这一点给出了说明。给定谱嵌入$Q \in \R^{n \times k},$如果用Lloyd求解,我们可以得到如下的矩阵表示
\begin{equation}
	\min_{G \in Ind,H\in \R^{k \times k}} \norm{Q - GH}_F^2
\end{equation}
如果用谱旋转便是
\begin{equation}
	\min_{G \in Ind,R^TR=RR^T=I_k} \norm{G-QR}_F^2
\end{equation}
根据酉不变性和正交矩阵对称阵也是正交矩阵可知,上式可以写作
\begin{equation}
	\min_{G \in Ind,R^TR=RR^T=I_k} \norm{Q-GR}_F^2
\end{equation}
经过对比,我们容易发现,之所以谱旋转比解$k$-means更好是因为对$H$施加了正交限制,缩小了$H$的范围自然更容易得到更好的$G$。

\subsection{带权kernel \texorpdfstring{$k$}{k}-means问题}
这个方法的基本思想是将谱聚类问题转为$k$-means问题,这样上一章所提出的各种有理论保证的算法就可以在谱聚类上。我们先介绍带权kernel $k$-means问题,再介绍这两者应该如何建立等价性。
\begin{definition}[带权kernel $k$-means问题]
    \label{def: weighted_kernel_kmeans}
    给定集合 $\mathcal{X} \subseteq \R^d$ 和每一个$x \in \mathcal{X}$ 的权重 $w_i$, 找到一个大小为$k$的集合 $C$ 使得下述目标函数能最小
    \begin{equation}
        \Psi(\mathcal{X},C) = \sum_{i=1}^k \sum_{x_j \in \pi_i}w_j \norm{\phi(x_j)-C_i}^2
    \end{equation}
    这里$\pi_i$是第$i$个类,$\cup_{i=1}^k \pi_i = \mathcal{X}$,$\phi(.)$表示映射函数,$C_i = \frac{\sum_{x_j \in \pi_i}w_j \phi(x_j)}{\sum_{j \in \pi_i}w_j}$。
\end{definition}
根据$K_{ij} = \phi(x_i)^T\phi(x_j)$,和上式$C_i$的定义,我们可以如下计算高维中点$\phi(a_i)$到一个中心点$m_c$的距离,
\begin{equation}
	\norm{\phi(a_i) - m_c}^2 = K_{i i}-\frac{2 \sum_{a_{j} \in \pi_{c}} w_{j} K_{i j}}{\sum_{a_{j} \in \pi_{c}} w_{j}}+\frac{\sum_{a_{j}, a_{l} \in \pi_{c}} w_{j} w_{l} K_{j l}}{\left(\sum_{a_{j} \in \pi_{c}} w_{j}\right)^{2}}
\end{equation}
由此,我们得到一个求解带权kernel $k$-means问题的算法,带权kernel Lloyd算法\citing{dhillon2004unified},见算法\ref{alg: weighted_kernel_Lloyd},该算法时间复杂度是$O(n^2 \tau)$,$\tau$是迭代次数。根据文献\cite{dhillon2004unified},带权重kernel $k$-means问题的目标函数值$\Psi(\mathcal{X},C)$可以写成矩阵形式
\begin{algorithm}
    \caption{带权kernel Lloyd算法}\label{alg: weighted_kernel_Lloyd}
    \KwIn{数据集$\mathcal{X}$,类数目$k$,kernel $K$,权重$w$,最大迭代次数$t_{\text{max}}$}
    \KwOut{$\mathcal{X}$的$k$个划分}
    随机将$\mathcal{X}$划分,记划分为$\pi$,当前的迭代次数为$t$,初始值0\\
    \While{$t<t_{\text{max}}$}{
    	对每个数据点$a_i$和类$c$的中心点$m_c$计算距离
	    \begin{equation*}
	    	d(a_i,m_c) \gets K_{i i}-\frac{2 \sum_{a_{j} \in \pi_{c}} w_{j} K_{i j}}{\sum_{a_{j} \in \pi_{c}} w_{j}}+\frac{\sum_{a_{j}, a_{l} \in \pi_{c}} w_{j} w_{l} K_{j l}}{\left(\sum_{a_{j} \in \pi_{c}} w_{j}\right)^{2}}
	    \end{equation*}\\
	    更新每个点对应的所属类,记点$a_i$更新后的类id是$C_i^* = \argmin\limits_{c = 1,2,...,k} d(a_i,m_c)$,则$\pi_c$更新后的划分是
	    \begin{equation*}
	    	\pi_c = \{a_i|C_i^*=c\}
	    \end{equation*}\\
	    $t\gets t+1$
    }
    \textbf{返回}$\mathcal{X}$的$k$个划分
\end{algorithm}
\begin{equation}
	% \label{eq: matrix_form_wkk}
	\Psi(\mathcal{X},C) = \operatorname{Tr}\left(W^{1 / 2} K W^{1 / 2}\right)-\operatorname{Tr}\left(\tilde{Y}^{T} W^{1 / 2} K W^{1 / 2} \tilde{Y}\right)
\end{equation}
其中$\tilde{Y} = W^{1/2}X(X^TWX)^{-1/2}$,这里$X\in B^{n\times k}$是上一小节的0/1划分矩阵,接下来,我们揭示谱聚类同上面介绍的带权kernel $k$-means问题的联系。回忆谱聚类是最小化Ncut,这也等价于最大化Nassoc,这里
\begin{equation}
	Nassoc=\sum_{i=1}^k \frac{\operatorname{linkage}(A_i,A_i)}{vol(A_i)} = \operatorname{Tr}(\tilde{Y}^TD^{-1/2}AD^{-1/2}\tilde{Y})
\end{equation}
其中$\tilde{Y} = D^{1/2}X(X^TDX)^{-1/2}$,这里我们取$W=D,K=D^{-1}AD^{-1}$,并注意到$Ncut+Nassoc=k$即可得到下式
\begin{equation}
	\label{eq: wkk_spectral_clustering}
	\Psi(\mathcal{X},C) = \operatorname{Tr}\left(D^{-1 / 2} A D^{-1 / 2}\right) - k + Ncut
\end{equation}
其中等式右边的前两项是一个和划分无关的常数,上式说明,只要令$W=D,K=D^{-1}AD^{-1}$,求解带权kernel $k$-means问题就是在解决谱聚类!通过前面的推导,我们知道带权$k$-means++是一个$O(\log k)$近似的算法,显然在kernel下依然有$O(\log k)$的近似,所以我们可以用$k$-means++或者其他的有理论保证的$k$-means算法使得谱聚类也有理论保证!不仅如此,这一等式还说明我们可以绕过谱分解从而减少大量时间。

除去上面的两种方法外还有其他的改进谱聚类聚类质量的方法,比如在relax的时候施加更紧的限制\citing{yang2018new},直接求解谱聚类的离散问题\citing{chen2018spectral}等等。

% 可以做的方向:coreset 谱聚类