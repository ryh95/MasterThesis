\section{本章小结}
本章我们先介绍了图论的背景,引入拉普拉斯矩阵,从图切割的角度引入了谱聚类。由于谱聚类需要做谱分解,而大规模数据下谱分解是非常耗时的,我们随后从三个方向上介绍了谱聚类的加速,第一个方法使用Nyström方法,这一方法是一种常用的矩阵低秩近似技术,它的副产品是加速特征向量的计算,通过低秩近似的相似度矩阵,得到了近似的拉普拉斯矩阵的特征向量,第二个方法是减少数据量,通过在一些采样集上做谱聚类从而得到所有点的划分,其中用Lloyd算法获得采样集的方式有理论上的道理,最后一个方法依然是用采样,不过采样的目的是构建低秩的相似度矩阵,这一矩阵的副产品是特征向量容易计算。然后,我们介绍了如何改进谱聚类的聚类质量,一种方法是改进谱分解后的聚类,算法是谱旋转,这一算法是贴合谱聚类的目标函数设计的,另一种方法是将谱分解同带权kernel $k$-means建立联系,用上一章所述的有理论保证的算法即可让谱聚类有理论保证。当然,也可直接用上一章的$k$-means算法替换这里谱分解后的Lloyd算法,这里不再赘述。接着,我们给出了基于采样的方法可以又快又好的完成谱聚类,并证明了这一方法可以有更紧的理论保证。在随后的实验环节中,我们在实验上验证了这一算法的有效性和高效性。

最后,我们指出一些谱聚类未来的可能研究方向:
\begin{enumerate}
	\item 作为谱聚类的核心,相似度矩阵的设计是一个开放性问题,不好的设计会让聚类结果和人的预期有较大出入。
	\item 用Coreset的思想加速谱聚类,使得谱聚类可以获得更好的理论结果。
	\item 解决均匀采样的一些问题,目前采样数和直径挂钩,所以采样数可能很大,另外,由于$K = D^{-1}AD^{-1}$,尽管只需要计算$n \times s$的kernel,但是需要完整计算相似度矩阵$A$,传统方法时间复杂度是$O(n^2)$,所以我们需要对这一步加速,并得出相应的理论结果。
\end{enumerate}
谱聚类作为非线性聚类的经典算法,早年被频繁的用在图像分割中,得益于其优秀的聚类效果,随着大数据时代的到来,新的分析技术和实用的算法必将给这一问题带来新的活力。